\documentclass[onecolumn, draftclsnofoot,10pt, compsoc]{IEEEtran}
\hbadness=1000 % suppress warnings
\usepackage{graphicx}
\usepackage{url}
\usepackage{setspace}
\usepackage{hyperref}
\usepackage{listings}
\usepackage{geometry}
\usepackage{caption}

\geometry{textheight=9.5in, textwidth=7in}

% 1. Fill in these details
\def \CapstoneTeamName{			}
\def \CapstoneTeamNumber{		69}
\def \GroupMemberOne{			Kin-Ho Lam}
\def \GroupMemberTwo{			Lucien Armand Tamdja Tamno}
\def \CapstoneProjectName{		Depth Sensing using Computer Vision and Lidar}
\def \CapstoneSponsorCompany{	Oregon State University}
\def \CapstoneSponsorPerson{	D. Kevin McGrath}


% 2. Uncomment the appropriate line below so that the document type works
\def \DocType{		%Problem Statement
	%Requirements Document
	%Technology Review
	%Design Document
	Winter Midterm Progress Report
}

\newcommand{\NameSigPair}[1]{\par
	\makebox[2.75in][r]{#1} \hfil 	\makebox[3.25in]{\makebox[2.25in]{\hrulefill} \hfill		\makebox[.75in]{\hrulefill}}
	\par\vspace{-12pt} \textit{\tiny\noindent
		\makebox[2.75in]{} \hfil		\makebox[3.25in]{\makebox[2.25in][r]{Signature} \hfill	\makebox[.75in][r]{Date}}}}
% 3. If the document is not to be signed, uncomment the RENEWcommand below
\renewcommand{\NameSigPair}[1]{#1}

%%%%%%%%%%%%%%%%%%%%%%%%%%%%%%%%%%%%%%%
\graphicspath{{images/}}
\begin{document}
	\begin{titlepage}
		\pagenumbering{gobble}
		\begin{singlespace}
			\centering
			\includegraphics[height=4cm,natwidth=345,natheight=435]{images/osu_logo.png}
			\hfill 
			% 4. If you have a logo, use this includegraphics command to put it on the coversheet.
			%\includegraphics[height=4cm]{CompanyLogo}   
			\par\vspace{.2in}
			\centering
			\scshape{
				\huge Senior Design Capstone \DocType \par
				{\large\today}\par
				\vspace{.5in}
				\textbf{\Huge\CapstoneProjectName}\par
				\vfill
				{\large Prepared for}\par
				\Huge \CapstoneSponsorCompany\par
				\vspace{5pt}
				{\Large\NameSigPair{\CapstoneSponsorPerson}\par}
				{\large Prepared by }\par
				Group\CapstoneTeamNumber\par
				% 5. comment out the line below this one if you do not wish to name your team
				\CapstoneTeamName\par 
				\vspace{5pt}
				{\large
					\NameSigPair{\GroupMemberOne}\par
					\NameSigPair{\GroupMemberTwo}\par
				}
				\vspace{20pt}
			}
			\begin{abstract}  
 				Depth Sensing with Computer Vision and LIDAR proposes combining computer vision and LIDAR to create a reliable depth sensor.
				This document details its project member's progress toward a final design, and future milestones.
			\end{abstract}     
		\end{singlespace}
	\end{titlepage}
\section{Table of Contents}
\tableofcontents
\bibliographystyle{IEEEtran}
\bibliography{ref}
\clearpage

\begin{singlespace}
	\section{Definitions}
		\subsection{IR}\label{def:IR}
		IR refers to the infrared light spectrum.

		\subsection{IR Depth Sensor}\label{def:depthsensor}
		A device that calculates distances by emitting infrared patterns. 
		
		\subsection{LIDAR}\label{def:lidar}
		Light Detection And Ranging - A method that uses lasers to measure distance
		
		\subsection{Microsoft Kinect}\label{def:kinect}
		A product that uses an IR Depth sensor to measure distances.
		
		\subsection{Logitech Brio Webcam}\label{def:brio}
		The webcam model this project shall be using.
		
		\subsection{RPLidar A1}\label{def:rplidar}
		A low-cost LIDAR unit that this project shall be using.
		
		\subsection{Computer Vision }\label{def:vision}
		The methods for acquiring, processing, analyzing, and classifying digital images and extracting information.
		
		\subsection{GUI}\label{def:gui}
		GUI: Graphical User Interface
		
		\subsection{Videocapture}\label{def:videocapture}
		videocapture: Stream of subsequent images
		
		
		
	\section{Project Purpose}
		Commercial infrared-based depth sensors such as the model used in Microsoft's Kinect can quickly calculate distances in indoor scenarios.
		However, IR depth sensors can be confused by other infrared emitters such as other IR depth sensors or natural sunlight.
		For these reasons, IR depth sensors cannot be used in self-driving cars, outdoor robots, or any any device that requires high accuracy and reliable distance calculation.

		Depth Sensing with Computer Vision and LIDAR proposes combining the power of computer vision with the reliability of LIDAR technology.
		LIDAR uses a pulsing laser to measure relative distance.
		The LIDAR unit we're going to be using is called the RPLidar A1.
		We'll be combining this with a high-end Logitech Brio webcam.

	\section{Current State}

	\subsection{Kin-Ho Lam}
	
	\subsection{Problems}
		
\end{singlespace}
\end{document}
