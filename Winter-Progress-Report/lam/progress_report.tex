\documentclass[onecolumn, draftclsnofoot,10pt, compsoc]{IEEEtran}
\hbadness=1000 % suppress warnings
\usepackage{graphicx}
\usepackage{url}
\usepackage{setspace}
\usepackage{hyperref}
\usepackage{listings}
\usepackage{geometry}
\usepackage{caption}

\geometry{textheight=9.5in, textwidth=7in}

% 1. Fill in these details
\def \CapstoneTeamName{			}
\def \CapstoneTeamNumber{		69}
\def \GroupMemberOne{			Kin-Ho Lam}
\def \CapstoneProjectName{		Depth Sensing using Computer Vision and Lidar}
\def \CapstoneSponsorCompany{	Oregon State University}
\def \CapstoneSponsorPerson{	D. Kevin McGrath}


% 2. Uncomment the appropriate line below so that the document type works
\def \DocType{
	Individual Winter Progress Report
}

\newcommand{\NameSigPair}[1]{\par
	\makebox[2.75in][r]{#1} \hfil 	\makebox[3.25in]{\makebox[2.25in]{\hrulefill} \hfill		\makebox[.75in]{\hrulefill}}
	\par\vspace{-12pt} \textit{\tiny\noindent
		\makebox[2.75in]{} \hfil		\makebox[3.25in]{\makebox[2.25in][r]{Signature} \hfill	\makebox[.75in][r]{Date}}}}
% 3. If the document is not to be signed, uncomment the RENEWcommand below
\renewcommand{\NameSigPair}[1]{#1}

%%%%%%%%%%%%%%%%%%%%%%%%%%%%%%%%%%%%%%%
\graphicspath{{images/}}
\begin{document}
	\begin{titlepage}
		\pagenumbering{gobble}
		\begin{singlespace}
			\centering
			\includegraphics[height=4cm,natwidth=345,natheight=435]{images/osu_logo.png}
			\hfill 
			% 4. If you have a logo, use this includegraphics command to put it on the coversheet.
			%\includegraphics[height=4cm]{CompanyLogo}   
			\par\vspace{.2in}
			\centering
			\scshape{
				\huge Senior Design Capstone \DocType \par
				{\large\today}\par
				\vspace{.5in}
				\textbf{\Huge\CapstoneProjectName}\par
				\vfill
				{\large Prepared for}\par
				\Huge \CapstoneSponsorCompany\par
				\vspace{5pt}
				{\Large\NameSigPair{\CapstoneSponsorPerson}\par}
				{\large Prepared by }\par
				Group\CapstoneTeamNumber\par
				% 5. comment out the line below this one if you do not wish to name your team
				\CapstoneTeamName\par 
				\vspace{5pt}
				{\large
					\NameSigPair{\GroupMemberOne}\par
				}
				\vspace{20pt}
			}
			\begin{abstract}  
 				Depth Sensing with Computer Vision and LIDAR proposes combining computer vision and LIDAR to create a reliable depth sensor.
				This document details its project member's progress toward a final design, and future milestones.
			\end{abstract}     
		\end{singlespace}
	\end{titlepage}
\section{Table of Contents}
\tableofcontents
\bibliographystyle{IEEEtran}
\bibliography{ref}
\clearpage

\begin{singlespace}
	\section{Definitions}
		\subsection{IR}\label{def:IR}
		IR refers to the infrared light spectrum.

		\subsection{IR Depth Sensor}\label{def:depthsensor}
		A device that calculates distances by emitting infrared patterns. 
		
		\subsection{LIDAR}\label{def:lidar}
		Light Detection And Ranging - A method that uses lasers to measure distance
		
		\subsection{Microsoft Kinect}\label{def:kinect}
		A product that uses an IR Depth sensor to measure distances.
		Referred to as a benchmark comparison for the purpose of this project.
		
		\subsection{Logitech Brio Webcam}\label{def:brio}
		Web-cam used for this project made by Logitech. \cite{logitech}
		
		\subsection{RPLidar A1}\label{def:rplidar}
		A budget LIDAR device used for this project made by Slamtec. \cite{slamtec}

		
		\subsection{Computer Vision }\label{def:vision}
		The methods for acquiring, processing, analyzing, and classifying digital images and extracting information.
		
		
	\section{Project Purpose}
		Commercial infrared-based depth sensors such as the model used in Microsoft's Kinect can quickly calculate distances in indoor scenarios.
		However, IR depth sensors can be confused by other infrared emitters such as other IR depth sensors or natural sunlight.
		For these reasons, IR depth sensors cannot be used in self-driving cars, outdoor robots, or any any device that requires high accuracy and reliable distance calculation.


		Depth Sensing with Computer Vision and LIDAR proposes combining the power of computer vision with the reliability of LIDAR technology.
		LIDAR uses a pulsing laser to measure relative distance.
		The LIDAR unit we're going to be using is called the RPLidar A1. \ref{rplidar}
		We'll be combining this with a high-end Logitech Brio Webcam. \ref{brio}

	\section{Current State}

	\subsection{Lam's Progress}
	
	\subsection{Partner Evaluation}
		My partner, Lucian Tamno, displays excellent work ethic, enthusiasm, dedication to our project, and an absolute commitment to do his best.
		I am very proud of my partner's performance, and I am confident we can achieve our project goals despite our late project start.
		Lucian's professional demeanor sets him apart and above many of our peers, as he is able to self-assign tasks and quickly comprehend technical concepts.
		I believe our individual experiences in professional working environments facilitates our team chemistry and communication.
		It is extremely refreshing to witness my project partner assume responsibilities and take ownership of his role.
		Where others may attribute their failures to others or blame external influences, Lucian displays level-headed and long-term thinking that will benefit him in his professional career.
		While we sometimes struggle with a small language barrier, I am certain this won't be an issue as we learn each other's mannerisms.
		I believe we have a lot to teach each other, and I look forward to our final term as we build this project to reflect our best efforts.
		
\end{singlespace}
\end{document}
